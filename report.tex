\documentclass{article}

\usepackage{graphicx}
\usepackage[brazil]{babel}
\usepackage{amsmath}
\usepackage{amssymb}
\usepackage{hyperref}
\usepackage{mathpazo}
\usepackage[margin=2cm]{geometry}
\usepackage{minted}
\usepackage{lmodern}

\title{CNPJ Insight - Relatório}
\author{Dávila Merieles \and Eduardo Adame \and Kayo Yokoyama \and Lucas Braga}
\date{\today}

\begin{document}
\maketitle

\section{Introdução}

O projeto CNPJ Insight tem como objetivo analisar os dados de empresas brasileiras disponibilizados pela Receita Federal. 

\section{Descrição do projeto}

\subsection{Estórias de usuário}

\subsection{Casos de uso}

\subsection{Diagrama de classes}

\subsection{Diagrama de pacote}

\subsection{Diagrama de sequência}

\subsection{Diagrama de componentes}

\section{Coleta de dados}

Os dados foram coletados através do site da Receita Federal, no endereço \url{https://dados.gov.br/dados/conjuntos-dados/cadastro-nacional-da-pessoa-juridica---cnpj}.

\section{Como executar}

Para executar o projeto, é necessário ter instalado o \texttt{docker} e o \texttt{docker-compose}. Para executar o projeto, basta executar o comando \texttt{docker-compose -f docker-compose.prod.yml up --build}, que irá construir e executar os containers necessários para o projeto. 

\section{Cobertura de testes}

\section{Documentação}

\section{Conclusão}

\end{document}